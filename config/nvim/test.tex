\documentclass{article}
\usepackage{amsmath}
\usepackage{amssymb}


% Custom commands
\newcommand{\abs}[1]{\left|#1\right|}
\newcommand{\dv}[2]{\frac{d#1}{d#2}y}
\newcommand{\dvOne}[1]{\frac{d}{d #1}}
\newcommand{\pdiv}[2]{\frac{\partial #1}{\partial #2}}
\newcommand{\pdivOne}[1]{\frac{\partial}{\partial #1}}
\newcommand{\pdivN}[3]{\frac{\partial^{#3} #1}{\partial #2^{#3}}}


\begin{document}
    Here is a different line
    And an even different line
    And now we want to add $\left(some parens\right))$
    now we're gonna test l( and look nothing but now

    $ \left(testing inside a paren\right)$
    now let's test some math:
    Suppose we consider a model such that

    \begin{equation*}
        y \sim \mathcal{N}(\mu, \sigma \vert \gamma, \beta, \Theta)
    \end{equation*}


    Now we'll add some new stuff 
    \begin{equation*}
        \begin{bmatrix}
            a &    b & 123234134434 & c \\
            4   &    5     &     \dots \\
        \end{bmatrix}
    \end{equation*}
    $b^{c} + f_{-1} a^{+}  R^{-}  a^{b} a^{c}$
    $ \frac{a}{b}$
    $ \frac{dx}{dt}$
    $\abs{x}, \abs{f(x)^{2}}$
    $\sqrt{4} + \sqrt{f(x) + 12 + \abs{14}}$
    $\frac{b^{2} \pm \sqrt{a^{2} + 4ac}}{2a}$
    $\binom{n}{k}$
    The partial derivative with respect to \( x \) is \(\pdivOne{x}\).
    The derivative with respect to $x$ is denoted $\dvOne{x}$

% More complex usage
The partial derivative with respect to \( x \) and \( y \) is \(\pdivOne{x} \pdivOne{y}\).
The first partial derivative of \( f \) with respect to \( x \) is \(\pdivN{f}{x}{1}\).

The second partial derivative of \( f \) with respect to \( x \) is \(\pdivN{f}{x}{2}\).

The mixed second partial derivative of \( f \) with respect to \( x \) and \( y \) is \(\pdivN{f}{x \partial y}{1}\).

 $\int_{a}^{b} f(x) \,dx$
 $test \int_{\infty}^{\infty}$
 $\equiv $
 $\int_{\infty}^{\infty} f(y) \,dy$


 Now we will write an integral:
 \begin{equation*}
     \int_{\infty}^{\infty}  \,dx
     \int  \,dx
     \int_{x}  \,dx
     \int_{[\infty, 0)}  \,dm \boxed{a}
     \iint_{[0,1] \times [0,1]}
 \end{equation*}                
 We will continue to test 

\end{document}
